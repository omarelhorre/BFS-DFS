\documentclass[12pt,a4paper]{article}
\usepackage[utf8]{inputenc}
\usepackage[T1]{fontenc}
\usepackage[french]{babel}
\usepackage{amsmath}
\usepackage{listings}
\usepackage{xcolor}
\usepackage{geometry}
\geometry{margin=1in}

% Configuration pour inclure les fichiers du répertoire courant
\makeatletter
\def\input@path{{./}}
\makeatother

% Code listing style
\lstset{
    language=C,
    basicstyle=\ttfamily\small,
    keywordstyle=\color{blue}\bfseries,
    commentstyle=\color{green!60!black},
    stringstyle=\color{red},
    numbers=left,
    numberstyle=\tiny\color{gray},
    stepnumber=1,
    numbersep=5pt,
    backgroundcolor=\color{gray!10},
    showspaces=false,
    showstringspaces=false,
    showtabs=false,
    frame=single,
    rulecolor=\color{black!30},
    tabsize=2,
    captionpos=b,
    breaklines=true,
    breakatwhitespace=false,
    escapeinside={\%*}{*)}
}

\title{Implémentation des Algorithmes de Parcours de Graphes BFS et DFS}
\author{}
\date{\today}

\begin{document}

\maketitle

\section{Introduction}

Les algorithmes de parcours de graphes sont des techniques fondamentales en informatique. Ce document présente une implémentation en C de deux algorithmes essentiels : la \textbf{Recherche en Profondeur (DFS)} et la \textbf{Recherche en Largeur (BFS)}. 

La DFS explore un graphe en profondeur avant de revenir en arrière, utilisant une pile (récursion). La BFS explore niveau par niveau, utilisant une file. Cette implémentation utilise une matrice d'adjacence pour représenter un graphe non orienté.

\section{Implémentation du Code}

Le code source complet se trouve dans le fichier \texttt{main.c}. Voici l'implémentation :

\lstinputlisting[caption={Implémentation complète des algorithmes BFS et DFS (main.c)}]{./main.c}

\section{Conclusion}

Cette implémentation démontre les algorithmes BFS et DFS utilisant une matrice d'adjacence. La DFS explore en profondeur via récursion, tandis que la BFS explore niveau par niveau via une file. Les deux algorithmes gèrent les graphes non connexes et fournissent une base solide pour comprendre les parcours de graphes.

\end{document}

